\documentclass[12pt]{article}                                              
\usepackage[papersize={8.5in,11in}]{geometry}
\usepackage[pdftex]{graphicx}
\DeclareGraphicsExtensions{.pdf,.png,.jpg}
%
\usepackage{amsmath}
\usepackage{amsthm}
\usepackage{amssymb}
\usepackage{textcomp}
\usepackage[all]{xy}
\usepackage{fancyhdr}
\usepackage{hyperref}
\usepackage{verbatim}
\usepackage{algorithm}
\usepackage{algorithmic}
\usepackage{color}
\usepackage[usenames,dvipsnames,svgnames,table]{xcolor}
\usepackage{rotating}
\usepackage{wrapfig}
\usepackage{tikz}
\usetikzlibrary{shapes.geometric, arrows}
\usepackage{framed}

\usepackage{listings}
\lstset{language=python,frame=ltrb,framesep=5pt,basicstyle=\normalsize,
 keywordstyle=\ttfamily\color{DarkRed},
%morecomment=[n][\textbf]{In\ [}{]\:},
%morecomment=[n][\textbf]{Out\ [}{]\:},
morecomment=[s][\color{blue}]{In\ [}{]\:},
morecomment=[s][\color{red}]{Out[}{]\:},
identifierstyle=\ttfamily\color{DarkBlue}\bfseries,
commentstyle=\color{OliveGreen},
stringstyle=\ttfamily,
showstringspaces=false,tabsize = 3}

\lstdefinelanguage{shell} {
commentstyle = \color{black},
keywordstyle = \color{black},
stringstyle = \color{black},
identifierstyle = \color{black},
morecomment=[s][\color{blue}]{In\ [}{]\:},
morecomment=[s][\color{red}]{Out[}{]\:},
 }


%
% this gives a little box for the end of a proof:
%
\def\endthrmbox{$\sqsubset \!\!\!\! \sqsupset$}

\newcommand{\dis}{\displaystyle}
 \def      \RR             {{\mathbb R}}
        \def      \NN             {{\Bbb N}}
        \def      \QQ             {{\Bbb Q}}
        \def      \CC             {{\Bbb C}}
        \def      \ZZ             {{\Bbb Z}}


        \def       \a              {{\alpha}}
        \def       \b              {{\beta}}
        \def       \d              {{\delta}}
        \def       \D              {{\Delta}}
        \def         \e              {{\varepsilon}}
        \def         \g              {{\gamma}}
        \def         \G              {{\Gamma}}
        \def       \l              {{\lambda}}
        \def       \L              {{\Lambda}}
        \def        \m               {{\mu}}
        \def         \n              {{\nabla}}
        \def       \var          {{\varphi}}
        \def         \s              {{\sigma}}
        \def       \Sig          {{\Sigma}}
        \def       \Om          {{\Omega}}

        \def       \t              {{\tau}}
        \def         \th             {{\theta}}
        \def       \O              {{\Omega}}
        \def       \o              {{\omega}}
        \def         \z              {{\zeta}}
       \def        \P             {{\Phi}}
       \def        \p             {{\phi}}
        %Other macros

        \def       \iy              {{\infty}}
        \def         \pa             {{\partial}}
        \def         \div           {{\rm div}}
         \def       \na            {{\nabla}}


%\renewcommand\baselinestretch{1.3}
%% The following block is for narrow margins:
\setlength{\topmargin}{-0.9in}
\setlength{\textheight}{9.50in}
\setlength{\oddsidemargin}{-0.375in}
\setlength{\evensidemargin}{-0.375in}
\setlength{\textwidth}{7.25in}
\setlength{\parindent}{0pt}
\setlength{\parskip}{2pt}
%% end page descript.




\usepackage[compact,explicit]{titlesec}
%\titleformat{\section}[runin]{\large\bfseries}{}{0pt}{\titlerule[1.5pt]\newline\vspace*{-4pt}
%Problem\quad\thesection\newline}[\vspace{0.01ex}{\titlerule[1.5pt]}]

\usepackage[export]{adjustbox}
\usepackage{float}
\usepackage{wrapfig}


\title{CSC445/545 - Exam \#2}
\author{Jim Leon}


\begin{document}
\maketitle
Due:  Apr 12, 2021 12am MST.
% Pre-problem text.  The preface to the homework.  Comments which apply to the homework overall.

%%\section{} %%  This will generate a numbered problem header, but you know this now.

\section{Program Description}
%This is the problem statement.
\paragraph{}
This program takes a file describing a Deterministic Finite Automata (DFA) or Deterministic 
Pushdown Automata (DPDA) and allows the user to check strings against it.  If the string is or 
isn't a part of the langauge on the machine, $L(M)$, the program will notify the user, 
respectively.  This program is run at the command line (terminal), and requires three arguments:  
the \textbf(main.py) executable, a flag \textbf(-F) or \textbf(-P) indicating what type of machine it is, 
and the file describing the machine.  To exit the program, the user can enter the string 'quit'.

\paragraph{}
This program requires Python3 or better be installed on your machine.  The program was developed 
on a Linux machine - running directly in a Windows terminal may require additional packages 
or set-up.

\section{Example of running the application}
\paragraph{}
To start up the application, the following must be provided at the command line (terminal):

\begin{verbatim}
$ ./main.py [-F, -P] machineFile
\end{verbatim}

\paragraph{}
Once the application fires up, you will see the following message:

\begin{verbatim}
Enter strings to check if they are in the language on the machine, L(M).
When you are finished, enter 'quit'
    
\end{verbatim}

\paragraph{}
From here the user can start entering strings - including the 'empty' string, which is the 
result of hitting the Enter key.  If the string (word) is in $L(M)$, you will be notified 
with the following message:

\begin{verbatim}
    00001 is in the language!
\end{verbatim}

...where, here, 00001 is a word in the language.  If a word is not in the language, however,
the terminal will notify:

\begin{verbatim}
    01010 is not in the language!
\end{verbatim}

When the user is ready to quit the application, they can enter the word 'quit'.  The terminal 
will notify and exit as follows:

\begin{verbatim}
    quit
    Goodbye!
$     
\end{verbatim}

\subsection{Structure of the Machine-description File}
\paragraph{}
The user must create a file that contains the description of the machine they wish to test 
with the application.  If the machine is a Finite Automata, it must contain three sets of 
information: the transitions on the machine, the starting state name, and a list of the final 
states.  Here is an example of a machine file describing $L = {a^nb: n >= 0}$:

\begin{verbatim}
    q0, a -> q0
    q0, b -> q1
    q1, a -> q2
    q1, b -> q2
    q2, a -> q2
    q2, b -> q2
    q0
    q1
\end{verbatim}

\paragraph{}
The first six lines are the transition functions, the seventh line is the starting state, and 
the last line is the final state of this machine.  In cases where your machine has several 
final states, you would list them as comma-separated values on the same line.

\paragraph{}
Here is an example of a Pushdown Automata machine description file:

\begin{verbatim}
    q0, a, 0 -> q1, 1q0, a, 0 -> q1, 10
    q1, a, 1 -> q1, 11
    q1, b, 1 -> q2,
    q2, b, 1 -> q2,
    q2,  , 0 -> q0,
    q0
    0
    q0
    0
    q0
\end{verbatim}

\paragraph{}
This machine describes $L = {a^nb^n, n >= 0}$.  The first five lines are the transition 
functions, the sixth line is the starting state name, the seventh line the starting symbol on 
the stack, and the last line the final state name.  Again, for multiple final states, you 
would separate the respective final states by commas.

\paragraph{}
Notice for the PDA that if you have transitions on empty strings '', you can simply leave out 
a character altogether (or insert a Space).  However, the comma is still required for proper 
parsing of the file.  In the Finite Automata machine descriptions you may also use empty 
('lambda') transitions using a similar rule.

\section{Code Structure and Primary Functions}
\paragraph{}
The program relies on two constructed classes, \textit{FiniteAutomata} and \textit{PushdownAutomata}, 
and one user interface set of functions as part of the \textit{ui.py} file.  
The \textit{ui.py} file methods do things like parsing the machine-description file and 'cleaning
up' the input.  It also handles the basic interaction between the user and the terminal.

\paragraph{}
From an end user's perspective the main program runs through the following
procedures/functions:

\begin{lstlisting}
import machine
import argparse
import ui

def main():
    if not args.FA and not args.PDA:
        print('Program requires three arguments to run')
        print('         ./main.py [-F,-P] [file]')
    elif args.FA:
        params = ui.fileParse(str(args.Machine),'f')
        ui.runNAProgram(params)
        print('Goodbye!')
    elif args.PDA:
        params = ui.fileParse(str(args.Machine),'p')
        ui.runPDAProgram(params)
        print('Goodbye!')
    return 0

main()
\end{lstlisting}

\paragraph{}
This excludes the \textit{argparse} functions needed for command line arguments;
however, the above code is essentially all the program requires.  The
\textit{ui.py} file contains the three functions, \textit{fileParse()}, \textit{runNAProgram()}, 
and \textit{runPDAProgram()}.

\paragraph{}
Listed below are those three functions.  As you can see, they invoke the \textit{FiniteAutomata} 
and \textit{PushdownAutomata} class objects, manipulate them, and run the program loop.  Once 
the loop is complete, control is handed back to the \textit{main()} function, and the 
program is ended.

\begin{lstlisting}
def fileParse(fileName: str, machType: str) -> list:
    fin = open(fileName, 'r')
    params = []
    if machType == 'f':
        fileRead = fin.readlines()
        transGraph = []
        if len(fileRead) < 3:
            raise Exception(InputExceptionMessage)
        for t in range(len(fileRead) - 2):
            transition = fileRead[t].strip()
            transGraph.append(buildTransFA(transition))
        params.append(transGraph)
        params.append(fileRead[len(fileRead)-2].strip())
        params.append(buildFinals(fileRead[len(fileRead)-1]))
    elif machType == 'p':
        fileRead = fin.readlines()
        transGraph = []
        if len(fileRead) < 4:
            raise Exception(InputExceptionMessage)
        for t in range(len(fileRead) - 3):
            transition = fileRead[t].strip()
            transGraph.append(buildTransPDA(transition))
        params.append(transGraph)
        params.append(fileRead[len(fileRead)-3].strip())
        params.append(fileRead[len(fileRead)-2].strip())
        params.append(buildFinals(fileRead[len(fileRead)-1]))
    fin.close()
    return params

def runNAProgram(params: list):
    print('Enter strings to check if they are in the language on the machine, L(M).')
    print('When you are finished, enter \'quit\'')
    M = machine.FiniteAutomata(params[0],params[1],params[2])
    userIn = input()
    while not userIn == 'quit':
        if M.run(userIn):
            print(userIn,'is in the language!')
        else:
            print(userIn,'is not in the language!')
        userIn = input()
    return 

def runPDAProgram(params: list):
    print('Enter strings to check if they are in the language on the machine, L(M).')
    print('When you are finished, enter \'quit\'')
    M = machine.PushdownAutomata(params[0],params[1],params[2],params[3])
    userIn = input()
    while not userIn == 'quit':
        if M.run(userIn):
            print(userIn,'is in the language!')
        else:
            print(userIn,'is not in the language!')
        userIn = input()
    return
\end{lstlisting}

\section{Program Limitations, Bugs, and To-Do's}
\paragraph{}
The biggest limitation to this application is it's deterministic nature.  Building nondeterministic 
machines is not possible based on the basic FA and PDA model, because the computer cannot 
accurately "guess" which move to take to guarantee word acceptance.  Therefore, although the 
user may certainly pass in any number of FA or PDA machine descriptions they choose, there is absolutely 
no guarantee that this application will be able to interpret if the word is in the language.

\paragraph{}
There should not be any major bugs in the program.  There is a suite of unit tests which can be 
run on the code.  There is always the chance that the machine definition file passed in may 
contain just the "right" configuration of symbols in it that the application may behave in 
unpredictable ways.  However, I was sure to include exception handling for as many foreseeable 
cases as I could.

\end{document}
